% Search for all the places that say "PUT SOMETHING HERE".

\documentclass[11pt]{article}
\usepackage{amsmath,textcomp,amssymb,geometry,graphicx,enumerate}

\def\Name{Dayne Tran}  % Your name
\def\SID{3034839991}  % Your student ID number
\def\Homework{00} % Number of Homework
\def\Session{Spring 2022 }


\title{EECS 16B --Spring 2022 --- Homework \Homework}
\author{\Name, SID \SID}
\markboth{EECS 16B--\Session\  Homework \Homework\ \Name}{EECS 16B--\Session\ Homework \Homework\ \Name}
\pagestyle{myheadings}
\date{}

\newenvironment{qparts}{\begin{enumerate}[{(}a{)}]}{\end{enumerate}}
\def\endproofmark{$\Box$}
\newenvironment{proof}{\par{\bf Proof}:}{\endproofmark\smallskip}

\textheight=9in
\textwidth=6.5in
\topmargin=-.75in
\oddsidemargin=0.25in
\evensidemargin=0.25in


\begin{document}
\maketitle

Collaborators: None

\section*{1. Policy Quiz}
The screenshot of my policy quiz results is attached at the end of this file.

\newpage
\section*{2.}
I understand how the Discord, Gradescope, and OH Queue tools work.

\newpage
\section*{3.}
\begin{qparts}
\item
We essentially want to ensure that $A$ is a positive semi-definite ma trix. So we condition that $det(A) = a^2 - b^2 > 0$.
\item
We know that $Ax = \lambda x$. For both eigenvaleus $\lambda_1$ and $\lambda_2$, we move $\lambda x$ over to the other side and take the determinant of both sides. Then, we take the system of equations and solve for $a$ and $b$.
\begin{align*}
    det(A - \frac{5}{2}) = 0\\
    det(A - \frac{9}{2}) = 0\\
    det\left(\begin{bmatrix}
        a-\frac{5}{2} & b \\
        b & a-\frac{5}{2}
    \end{bmatrix}\right) = 0\\
    det\left(\begin{bmatrix}
        a-\frac{9}{2} & b \\
        b & a-\frac{9}{2}
    \end{bmatrix}\right) = 0\\
    \left(a-\frac{5}{2}\right)^2 - b^2 = 0\\
    \left(a-\frac{9}{2}\right)^2 - b^2 = 0\\
    a^2 - \frac{20}{4}a+\frac{25}{4} - b^2 = 0\\
    a^2 - \frac{36}{4}a+\frac{81}{4} - b^2 = 0\\
    \frac{16}{4}a-\frac{56}{4} = 0\\
    a = \frac{7}{2}\\
    b=1
\end{align*}
\item
Solve for the eigenvalues, then normalized eigenvectors, of $\widehat{H}$. 
\begin{align*}
    det(A - \lambda I) = 0\\
    (3-\lambda)^2 - 2^2 = 0\\
    \lambda^2-6\lambda + 5 = 0\\
    \lambda_1 = 1, \lambda_2 = 5\\
    A-1I = 0 \Longrightarrow \vec{v}_{\lambda_1} = \begin{bmatrix}\frac{\sqrt{2}}{2}\\-t6t\frac{\sqrt{2}}{2}\end{bmatrix}\\
    A-5I = 0 \Longrightarrow \vec{v}_{\lambda_2} = \begin{bmatrix}\frac{\sqrt{2}}{2}\\\frac{\sqrt{2}}{2}\end{bmatrix}\\
\end{align*}
Solve the system of equations $\vec{v}_s = \alpha \vec{v}_{\lambda_1} + \beta  \vec{v}_{\lambda_2}$ for $\alpha$ and $\beta$. Then find the magnitude of $\alpha$.
\begin{align*}
    0 = \alpha \frac{\sqrt{2}}{2} + \beta \frac{\sqrt{2}}{2} \\
    1 = - \alpha \frac{\sqrt{2}}{2} + \beta \frac{\sqrt{2}}{2} \\
    1 = \beta \sqrt{2} \\
    \beta = \frac{\sqrt{2}}{2}, \alpha = -\frac{\sqrt{2}}{2} \\
    |\alpha| = \frac{\sqrt{2}}{2}
\end{align*}
\end{qparts}


\newpage
\section*{4.}
\begin{qparts}
\item
First, we test for symmetry.
\begin{align*}
    \begin{bmatrix}2\\1\end{bmatrix}^T (3\vec{x} + 3\vec{y}) = 6\vec{x} + 3\vec{y} = {(3\vec{x} + 3\vec{y})}^T \begin{bmatrix}2\\1\end{bmatrix}
\end{align*}
Second, we test for linearity.
\begin{align*}
    \langle3\vec{x} + 3\vec{y}, \begin{bmatrix}2\\1\end{bmatrix}\rangle =
    3\langle\vec{x}, \begin{bmatrix}2\\1\end{bmatrix}\rangle + 3\langle\vec{y}, \begin{bmatrix}2\\1\end{bmatrix}\rangle \\
    = 3(2\vec{x})
\end{align*}
\end{qparts}

\newpage
\section*{5.}
YOUR ANSWER GOES HERE

\end{document}
% Search for all the places that say "PUT SOMETHING HERE".

\documentclass[11pt]{article}
\usepackage{amsmath,textcomp,amssymb,geometry,graphicx,enumerate}

\def\Name{Dayne Tran}  % Your name
\def\SID{3034839991}  % Your student ID number
\def\Homework{00} % Number of Homework
\def\Session{Spring 2022 }


\title{EECS 16B --Spring 2022 --- Homework \Homework}
\author{\Name, SID \SID}
\markboth{EECS 16B--\Session\  Homework \Homework\ \Name}{EECS 16B--\Session\ Homework \Homework\ \Name}
\pagestyle{myheadings}
\date{}

\newenvironment{qparts}{\begin{enumerate}[{(}a{)}]}{\end{enumerate}}
\def\endproofmark{$\Box$}
\newenvironment{proof}{\par{\bf Proof}:}{\endproofmark\smallskip}

\textheight=9in
\textwidth=6.5in
\topmargin=-.75in
\oddsidemargin=0.25in
\evensidemargin=0.25in


\begin{document}
\maketitle

Collaborators: None

\section*{1. Policy Quiz}
The screenshot of my policy quiz results is attached at the end of this file.

\newpage
\section*{2.}
I understand how the Discord, Gradescope, and OH Queue tools work.

\newpage
\section*{3.}
\begin{qparts}
\item
We essentially want to ensure that $A$ is a positive semi-definite ma trix. So we condition that $det(A) = a^2 - b^2 > 0$.
\item
We know that $Ax = \lambda x$. For both eigenvaleus $\lambda_1$ and $\lambda_2$, we move $\lambda x$ over to the other side and take the determinant of both sides. Then, we take the system of equations and solve for $a$ and $b$.
\begin{align*}
    det(A - \frac{5}{2}) = 0\\
    det(A - \frac{9}{2}) = 0\\
    det\left(\begin{bmatrix}
        a-\frac{5}{2} & b \\
        b & a-\frac{5}{2}
    \end{bmatrix}\right) = 0\\
    det\left(\begin{bmatrix}
        a-\frac{9}{2} & b \\
        b & a-\frac{9}{2}
    \end{bmatrix}\right) = 0\\
    \left(a-\frac{5}{2}\right)^2 - b^2 = 0\\
    \left(a-\frac{9}{2}\right)^2 - b^2 = 0\\
    a^2 - \frac{20}{4}a+\frac{25}{4} - b^2 = 0\\
    a^2 - \frac{36}{4}a+\frac{81}{4} - b^2 = 0\\
    \frac{16}{4}a-\frac{56}{4} = 0\\
    a = \frac{7}{2}\\
    b=1
\end{align*}
\item
Solve for the eigenvalues, then normalized eigenvectors, of $\widehat{H}$. 
\begin{align*}
    det(A - \lambda I) = 0\\
    (3-\lambda)^2 - 2^2 = 0\\
    \lambda^2-6\lambda + 5 = 0\\
    \lambda_1 = 1, \lambda_2 = 5\\
    A-1I = 0 \Longrightarrow \vec{v}_{\lambda_1} = \begin{bmatrix}\frac{\sqrt{2}}{2}\\-t6t\frac{\sqrt{2}}{2}\end{bmatrix}\\
    A-5I = 0 \Longrightarrow \vec{v}_{\lambda_2} = \begin{bmatrix}\frac{\sqrt{2}}{2}\\\frac{\sqrt{2}}{2}\end{bmatrix}\\
\end{align*}
Solve the system of equations $\vec{v}_s = \alpha \vec{v}_{\lambda_1} + \beta  \vec{v}_{\lambda_2}$ for $\alpha$ and $\beta$. Then find the magnitude of $\alpha$.
\begin{align*}
    0 = \alpha \frac{\sqrt{2}}{2} + \beta \frac{\sqrt{2}}{2} \\
    1 = - \alpha \frac{\sqrt{2}}{2} + \beta \frac{\sqrt{2}}{2} \\
    1 = \beta \sqrt{2} \\
    \beta = \frac{\sqrt{2}}{2}, \alpha = -\frac{\sqrt{2}}{2} \\
    |\alpha| = \frac{\sqrt{2}}{2}
\end{align*}
\end{qparts}


\newpage
\section*{4.}
\begin{qparts}
\item
First, we test for symmetry.
\begin{align*}
    \begin{bmatrix}2\\1\end{bmatrix}^T (3\vec{x} + 3\vec{y}) = 6x_1+3x_2 + 6y_1+3y_2 = {(3\vec{x} + 3\vec{y})}^T \begin{bmatrix}2\\1\end{bmatrix}
\end{align*}
Second, we test for linearity.
\begin{align*}
    \langle3\vec{x} + 3\vec{y}, \begin{bmatrix}2\\1\end{bmatrix}\rangle = 6x_1+3x_2 + 6y_1+3y_2 =
    3\langle\vec{x}, \begin{bmatrix}2\\1\end{bmatrix}\rangle + 3\langle\vec{y}, \begin{bmatrix}2\\1\end{bmatrix}\rangle
\end{align*}
Third, we test for positive semi-definiteness.
\begin{align*}
    \langle \begin{bmatrix}2\\1\end{bmatrix}, \begin{bmatrix}2\\1\end{bmatrix}\rangle &= 5 > 0& \\
    \langle 3x+3y, 3x + 3y \rangle 
    &= \langle 3 \begin{bmatrix}x_1 + y_1\\x_2+y_2\end{bmatrix}, 3 \begin{bmatrix}x_1 + y_1\\x_2+y_2\end{bmatrix}\rangle 
    = 9(x_1+y_1)^2 + 9(x_2+y_2)^2 > 0&
\end{align*}
All three innner product properties hold.

\item
$2 \cdot 2 + 5 \cdot 5 + 6 \cdot 6 + 2 \cdot 2 = 69$
\end{qparts}

\newpage
\section*{5.}
\begin{qparts}
\item
Projecting $\vec{x}$ onto $\vec{y}$ means the following:
\begin{align*}
    proj_{\vec{y}}\vec{x} = \frac{\langle \vec{y}, \vec{x} \rangle}{\langle \vec{y}, \vec{y} \rangle}\vec{y}
\end{align*}
Project $\vec{x}_{sample}$ onto the footprint of Electric Love, $\vec{x}_1$:
\begin{align*}
    proj_{\vec{x_1}}\vec{x}_{sample} = \frac{\langle \vec{x}_1, \vec{x}_{sample} \rangle}{\langle \vec{x}_1, \vec{x}_1 \rangle} \vec{x}_1 =
    \frac{(1 \cdot 2) + (-1 \cdot 0) + (1 \cdot -1) + (-1 \cdot 1)}{(1 \cdot 1) + (-1 \cdot -1) + (1 \cdot 1) + (-1 \cdot -1)} \vec{x}_1= 
    \frac{0}{4}\vec{x}_1 = \vec{0}
\end{align*}
Project $\vec{x}_{sample}$ onto the footprint of She's Electric, $\vec{x}_2$:
\begin{align*}
    proj_{\vec{x_2}}\vec{x}_{sample} = \frac{\langle \vec{x}_2, \vec{x}_{sample} \rangle}{\langle \vec{x}_2, \vec{x}_2 \rangle} \vec{x}_2 =
    \frac{(2 \cdot 2) + (-2 \cdot 0) + (-8 \cdot -1) + (7 \cdot 1)}{(2 \cdot 2) + (-2 \cdot -2) + (-8 \cdot -8) + (7 \cdot 7)} \vec{x}_2= 
    \frac{19}{128}\vec{x}_2
\end{align*}
Project $\vec{x}_{sample}$ onto the footprint of Electric Feel, $\vec{x}_3$:
\begin{align*}
    proj_{\vec{x_3}}\vec{x}_{sample} = \frac{\langle \vec{x}_3, \vec{x}_{sample} \rangle}{\langle \vec{x}_3, \vec{x}_3 \rangle} \vec{x}_3 =
    \frac{(4 \cdot 2) + (1 \cdot 0) + (-2 \cdot -1) + (2 \cdot 1)}{(4 \cdot 4) + (1 \cdot 1) + (-2 \cdot -2) + (2 \cdot 2)} \vec{x}_3= 
    \frac{12}{25}\vec{x}_3
\end{align*}
Now, we find the error $\vec{e} = \vec{x}_{sample} - proj$.
\begin{align*}
    e_1 &= \vec{x}_{sample} - proj_{\vec{x_1}}\vec{x}_{sample} = \begin{bmatrix}2&0&-1&1\end{bmatrix}^T& \\
    e_2 &= \vec{x}_{sample} - proj_{\vec{x_2}}\vec{x}_{sample} = \begin{bmatrix}0.297&0& -0.148& 0.148\end{bmatrix}^T& \\
    e_3 &= \vec{x}_{sample} - proj_{\vec{x_3}}\vec{x}_{sample} = \begin{bmatrix}0.96&0& -0.48& 0.48\end{bmatrix}^T&
\end{align*}
Now, we find the error vector which has the smallest magnitude; the song that corresponds with this minimum-difference vector is likeliest song of the three to be playing according to the sample footprint.
\begin{align*}
    ||e_1|| = 2.45 \\
    ||e_2|| = 0.36 \\
    ||e_3|| = 1.48
\end{align*}
We conclude that the song playing must be Electric Feel.
\item
The cross-correlation plot is highest at the 180-second mark, so we believe that to be when the sample was taken.
\item
(ii) $\vec{a}_n^T (M M^T)^{-1} M \vec{b}_n$
\item
No.
\item
Three.
\end{qparts}

\newpage
\section*{6.}
\begin{qparts}
\item
Gaussian: 6-second shift is $6s \cdot \frac{300 m}{s} = 1800$ m away. NVA: 300 m away. Charge: 900 m away.
\item
$\vec{b}$ aligns most closely with the distances.
\item
We can find our location ($x_1, x_2$) using the following system of equations:
\begin{align*}
    \begin{pmatrix}
        2 \frac{\vec{a}_1^T}{v\tau_1} - 2 \frac{\vec{a}_2^T}{v\tau_2} \\
        2 \frac{\vec{a}_1^T}{v\tau_1} - 2 \frac{\vec{a}_3^T}{v\tau_3}
    \end{pmatrix}
    \begin{pmatrix}
        x_1 \\
        x_2
    \end{pmatrix}
    = 
    \begin{pmatrix}
        v \tau_2 - v \tau_1 + \frac{\vec{{a}_1^T}\vec{a}_1}{v\tau_1} - \frac{\vec{{a}_2^T}\vec{a}_2}{v\tau_2} \\
        v \tau_3 - v \tau_1 + \frac{\vec{{a}_1^T}\vec{a}_1}{v\tau_1} - \frac{\vec{{a}_3^T}\vec{a}_3}{v\tau_3}
    \end{pmatrix}
\end{align*}
where 1 corresponds to the Gaussian tower, 2 corresponds to the NVA tower, and 3 corresponds to the Charge tower. 
After plugging everything in and doing arithmetic, we arrive at the two following equations:
\begin{align*}
    \frac{32}{\sqrt{72}}x_1 + \frac{22}{\sqrt{72}}x_2 = \sqrt{72} - \sqrt{18}+\frac{53}{\sqrt{18}}-\frac{53}{\sqrt{72}} \\
    \frac{14}{\sqrt{18}}x_1 + (\frac{4}{\sqrt{18}} - \frac{18}{116})x_2 = \sqrt{116} - \sqrt{18} + \frac{53}{\sqrt{18}} - \frac{81}{\sqrt{116}}
\end{align*}
We find that our coordinates are (3.314, -0.775).
\end{qparts}

\newpage
\section*{7.}
\begin{qparts}
\item

\end{qparts}

\newpage
\section*{8.}
\begin{qparts}
\item
iii. The resistive plate can be placed in either location.
\item
$\frac{150 \Omega}{R_1 + 150 \Omega} \cdot 10 V = 5V \Longrightarrow R_1 = 150 \Omega$
\item
We apply the first golden rule, $I_+ = I_- = 0$.
\begin{align*}
    I_- &= I_R + I_in = 0 \Longrightarrow I_R = -I_{in}&
    \intertext{Now we apply the second golden rule, $V_+ = V_-$.}
    V_{out} - V_{-} &=i_R \cdot R(t)& \\
    V_+ &= V_{REF} = V_{out}-i_R \cdot R(t) = V_-& \\
    V_{REF} &= V_{out} - (-I_{in}) \cdot R(t)& \\
    &= 0 - (-4mA) \cdot 2.5& \\
    &= 10 mV&
\end{align*}
\item
$u_B \rightarrow 4, u_A \rightarrow 7, 5 V \rightarrow 3, -5 V \rightarrow 2, V_{out} \rightarrow 6$.
\end{qparts}

\newpage
\section*{9.}
\begin{qparts}
\item

\end{qparts}
\newpage
\section*{11.}
\begin{qparts}
\item
$U_-$ decreases.
\item
$U_+$ increases.
\item
$V_{error}$ increases.
\item
$V_x$ increases.
\item
$V_{out}$ decreases.
\item
The circuit is in positive feedback.
\end{qparts}
 
\newpage
\section*{12.}
\begin{qparts}
\item

\end{qparts}
\newpage
\section*{13.}
First band means most significant digit, so \textit{green} for 50.
Second band means second most significantI, so \textit{brown} for 1.
Third band means multiplier, so we want \textit{black} for 1x51 = 51.
Fourth band means tolerance, which we will make gold for $5\%$.
\newpage
\section*{14.}
\begin{qparts}
\item
\begin{align*}\langle f, g \rangle &= f(0)g(0) + f(1)g(1) + f(2)g(2)& 
    \\ &= (1 \cdot -1) + (0 \cdot 0) + (1 \cdot -1)& \\
    & = -2& \end{align*}
$-x^2 + 2x - 1$  is not a possible expression for $g(x)$.
\item
\begin{align*}\langle f, g \rangle = (1 \cdot -1) + (0 \cdot 1) + (1 \cdot 5) = 4\end{align*}
$x^2 + x - 1$ is not a possible expression for $g(x)$.
\item
\begin{align*}\langle f, g \rangle = (1 \cdot -1) + (0 \cdot 0) + (1 \cdot 1) = 0\end{align*}
$x-1$ is a viable expression for $g(x)$.
\item
\begin{align*}\langle f, g \rangle = (1 \cdot 0) + (0 \cdot 1) + (1 \cdot 2) = 2\end{align*}
$x$ is not a possible expression for $g(x)$.
\end{qparts}

\newpage
\section*{15.}
$\vec{v}_2$ is best suited as a satellite code; it has just one distinct instance of a different value, and it also is not an indicator of some pattern.

\newpage
\section*{16.}
We need to null out 1 source.
\newpage
\section*{17.}
\begin{qparts}
\item
$\vec{x}[1] = [0.5, 0.25, 0.25]^T$
\item
$
    \begin{pmatrix}
        \frac{1}{2} & \frac{1}{2} & 0 \\
        \frac{1}{4} & \frac{1}{4} & \frac{1}{2} \\
        \frac{1}{4} & \frac{1}{4} & \frac{1}{2}
    \end{pmatrix} 
$
\item
$[0, \frac{1}{2}, \frac{1}{2}]$
\end{qparts}
\newpage
\section*{18.}
\newpage
\section*{19.}
Hell nah, fam.
\newpage
\section*{20.}
\begin{qparts}
\item
I used the course notes from EECS 16A to help myself remember and understand concepts in this problem set.
\item
I did not work with anyone else on this problem set.
\item
I worked roughly 10 total hours on the problem set.
\end{qparts}
\end{document}
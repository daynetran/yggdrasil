\documentclass[10pt,a4paper]{article}
\usepackage[utf8]{inputenc}
\usepackage{amsmath}
\makeatletter
\renewcommand*\env@matrix[1][*\c@MaxMatrixCols c]{%
  \hskip -\arraycolsep
  \let\@ifnextchar\new@ifnextchar
  \array{#1}}
\makeatother
\usepackage{amsfonts}
\usepackage{amssymb}
\begin{document}
\section{Reading ssignment}
\section{Filtering Out The Troll}
\begin{itemize}
  \item[(a)]
    \begin{align*}
      \vec{m}_{1} &= \text{cos}(\alpha) \cdot \vec{s} + \text{cos}(\beta) \cdot \vec{r}&\\
      &= \text{cos}(\frac{\pi}{4}) \cdot \vec{s} + \text{cos}(-\frac{\pi}{6}) \cdot \vec{r}&\\
      &= \frac{\sqrt{2}}{2} \cdot \vec{s} + \frac{\sqrt{3}}{2} \cdot \vec{r}&\\
      \vec{m}_{2} &= \text{sin}(\alpha) \cdot \vec{s} + \text{sin}(\beta) \cdot \vec{r}& \\
      &= \text{sin}(\frac{\pi}{4}) \cdot \vec{s} + \text{sin}(-\frac{\pi}{6}) \cdot \vec{r}&\\
      &= \frac{\sqrt{2}}{2} \cdot \vec{s} - \frac{1}{2} \cdot \vec{r}&\\
    \end{align*}
  \item[(b)] We can substitute through $\vec{r}$ to get $\vec{s}$ as a weighted combination of $\vec{m}_1$ and $\vec{m}_2$.
    \begin{align*}
      \vec{r} &= \frac{2}{\sqrt{3}}\vec{m}_{1} - \frac{\sqrt{2}}{\sqrt{3}} \cdot \vec{s}& \\
      \vec{m}_{2} &= \frac{\sqrt{2}}{2} \cdot \vec{s} - \frac{1}{2} \cdot (\frac{2}{\sqrt{3}}\vec{m}_{1} - \frac{\sqrt{2}}{\sqrt{3}} \cdot \vec{s})& \\
      \frac{\sqrt{6}+\sqrt{2}}{2\sqrt{3}} \cdot \vec{s} &= -\frac{1}{\sqrt{3}}\vec{m}_{1} - \vec{m}_{2}&
    \end{align*}
\end{itemize}

\section{Multiply the Matrices}
\begin{itemize}
  \item[(a)] Yes, it is a valid operation; the dimensions of \textbf{AB} is $3 \times 4$.

  \item[(b)]
    \begin{align*}
      AB &=
        \begin{bmatrix}
          1 \cdot 1 + 0 \cdot -3 & 1 \cdot 2 + 0 \cdot 0 & 1 \cdot -1 + 0 \cdot 2 & 1 \cdot 0 + 0 \cdot -1 \\
          2 \cdot 1 + 1 \cdot -3 & 2 \cdot 2 + 1 \cdot 0 & 2 \cdot -1 + 1 \cdot 2 & 2 \cdot 0 + 1 \cdot -1 \\
          0\cdot 1 + 1\cdot -3 & 0\cdot 2 + 1\cdot 0 & 0\cdot -1 + 1\cdot 2 & 0\cdot 0 + 1\cdot -1
        \end{bmatrix} \\
        &=
        \begin{bmatrix}
          1 & 2 & -1 & 0 \\
          -1 & 4 & 0 & -1 \\
          -3 & 0 & 2 & -1
        \end{bmatrix} \\
    \end{align*}
    \textbf{BA} is invalid because the middle dimensions (4 and 3) are not equal.

  \item[(c)] goo goo ga ga
\end{itemize}

\section{Linear Dependence}
\begin{itemize}
  \item[(a)] The set of vectors is linearly independent.

  \item[(b)] The set of vectors is linearly dependent: $-2\vec{v_1} + \vec{v_2} + 3\vec{v_3}$

  \item[(c)]
  \begin{align*}
    \begin{pmatrix}[ccc|c]
        2 & 0 & 2 & 0 \\
        2 & 1 & 4 & -1 \\
        0 & 1 & -1 & 1
       \end{pmatrix}
    &= \begin{pmatrix}[ccc|c]
        1 & 0 & 1 & 0 \\
        0 & -1 & -2 & 1 \\
        0 & 1 & -1 & 1
       \end{pmatrix} & \\
    &= \begin{pmatrix}[ccc|c]
        1 & 0 & 1 & 0 \\
        0 & 1 & -1 & 1 \\
        0 & 0 & -3 & 2
       \end{pmatrix} & \\
    &= \begin{pmatrix}[ccc|c]
        1 & 0 & 0 & \frac{2}{3} \\
        0 & 1 & 0 & \frac{1}{3} \\
        0 & 0 & 1 & -\frac{2}{3}
       \end{pmatrix}
  \end{align*}
  The set is linearly dependent because the last vector is a linear combination of the first three vectors: $\frac{2}{3}\vec{v_1} + \frac{1}{3}\vec{v_2} +
  -\frac{2}{3}\vec{v_3} - \vec{v_4} = \mathbf{\vec{0}}$

  \item[(d)] The set is linearly independent.
\end{itemize}

\section{Linear Dependence in a Square Matrix}

\section{Image Stitching}
\begin{itemize}
  \item[(a)] \textbf{Step 1:} The two geometric transformations that get applied to $\vec{u}$ to get $\vec{v_1}$ are rotation (we set a new basis with the $2 \times 2$ matrix) and scaling (there is a factor of 2 that can drawn out of the matrix). \textbf{Step 2:} The addition of $\vec{w}$ applies translation/shifting to $\vec{v_1}$.
\end{itemize}
\end{document}

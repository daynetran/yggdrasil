\documentclass[10pt,a4paper]{article}
\usepackage[utf8]{inputenc}
\usepackage{amsmath}
\makeatletter
\renewcommand*\env@matrix[1][*\c@MaxMatrixCols c]{%
  \hskip -\arraycolsep
  \let\@ifnextchar\new@ifnextchar
  \array{#1}}
\makeatother
\usepackage{amsfonts}
\usepackage{amssymb}
\begin{document}
\section{Probability Potpourri}
\begin{itemize}
  \item[1.] We want to prove the covariance matrix $\Sigma$ is always positive semi-definite, i.e. $a^{T}\Sigma a >= 0$ for all $n \times 1$ vectors $a$. Given $\Sigma = E[(Z - \mu)(Z - \mu)^{T}]$, set the following equation:
  \begin{align*}
    a^{T}E[(Z - \mu)(Z - \mu)^{T}] a >= 0 \\
    E[a^{T}(Z - \mu)(Z - \mu)^{T}a] >= 0 \intertext{by linearity of expectation}
    E[((Z-\mu)a)^{T}((Z-\mu)a)] >= 0 \\
    E[||(Z - \mu)a||^{2}] >= 0
    \end{align*}

  \item[2.] Let $W$ represent the event that there is a gust of wind and let $H$ represent the event that the archer hits her target. We know that $P(H|W) = 0.4$, $P(H | W^C) = 0.7$, and $P(W) = 0.3$.
    \begin{enumerate}
      \item P(there is gust of wind and she hits target)
            = $P(W \cap H) \\
            = P(W) * P(H | W) \\
            = 0.3 * 0.4 \\
            = 0.12$
      \item P(hits the target on the first shot) \\
            $= P(H) \\
            = P(H \cap W) + P(H \cap W^C) \\
            = 0.12 + P(W) * P (H | W^C) \\
            = 0.12 + 0.7*0.7 \\
            = 0.62$
      \item P(hits the target exactly once in two shots) \\
            =$P(H_{1} \cap H^C_{2}) + P(H^{C}_{1} \cap H_{2}) \\
            =(0.62)(1-0.62) + (1-0.62)(0.62) \\
            =0.4712$
      \item P(there was no gust of wind on an occasion when she missed) \\
            =$P(W^C | H^C) \\
            =\frac{P(W^C \cap H^C)}{P(H^C)} \\
            =\frac{P(W^C)P(H^C | W^C)}{P(H^C)} \\
            =\frac{0.7 * 0.3}{1-0.62} \\
            =0.55$
    \end{enumerate}

  \item[3.] Let $Y$ represent the score of a single strike, or $Y = g(X=x)$ such that
  \begin{equation*}
    g(X=x) = \begin{cases}
            4, & \text{if}~ x\leq \frac{1}{\sqrt{3}} \\
            3, & \text{if}~ \frac{1}{\sqrt{3}} < x \leq 1 \\
            2, & \text{if}~ 1 < x \leq \sqrt{3} \\
            0, & \text{otherwise.}
           \end{cases}
  \end{equation*}
  Find expectation $E[Y] = \int yf(y)dy = \int g(x)f(x)dx = \int_{0}^{\infty}g(x)\frac{2}{\pi (1+x^2)}dx$
  \begin{align*}
    &=4 * \frac{2}{\pi}arctan(x) \mid_{0}^{\frac{1}{\sqrt{3}}} + 3 * \frac{2}{\pi}arctan(x) \mid_{\frac{1}{\sqrt{3}}}^{1} + 2 * \frac{2}{\pi}arctan(x) \mid_{1}^{\sqrt{3}}& \\
    &=4*\frac{2}{\pi}(\frac{\pi}{6} - 0) + 3*\frac{2}{\pi}(\frac{\pi}{4}-\frac{\pi}{6}) + 2*\frac{2}{\pi}(\frac{\pi}{3} - \frac{\pi}{4})& \\
    &= 2.167&
  \end{align*}
\end{itemize}
\pagebreak

\section{Properties of Gaussians}
\begin{itemize}
  \item[1.] We want to prove that $E[e^{\lambda X}] = e^{\sigma^{2}\lambda{2}/2}$
\end{itemize}
\pagebreak

\section{Linear Algebra Review}
\begin{itemize}
  \item[1.] We will prove equivalence between these three different definitions of PSD using a cycle. First, we will prove that given (b), (b) $\implies$ (a).
  \begin{align*}
    x^{T}Ax = x^{T}\lambda x = \lambda x^{T}x = \lambda |x|_2 \geq 0 \\
    \intertext{We have shown that given A has nonnegative eigenvalues (from (b)), we maintain the inequality given in (a).}
  \end{align*}
  Second, we will prove that given (a), (a) $\implies$ (c).
  \begin{align*}
    x^{T}UU^{T}x = (U^{T}x)^{T}U^{T}x = |U^{T}x|_2 \geq 0
    \intertext{We have shown that given the inequality in (a), we can find a matrix U from (c) that upholds (a).}
  \end{align*}
  Third, we will prove that given (c), (c) $\implies$ (b).
  \begin{align*}
    \intertext{A is a symmetric matrix}
  \end{align*}
\end{itemize}

\section{Gradients and Norms}
\begin{itemize}
  \item[1.] First, we will prove that $\frac{1}{\sqrt{n}}||x||_{2} \leq ||x||_\infty$. We know that $||x||_{\infty} = max(x)$. Let's use the Cauchy-Schwartz inequality to produce an upper bound for $\frac{1}{\sqrt{n}}||x||_{2}$. Let $\mathbf{1}$ be the ones $n \times 1$ vector. The Cauchy-Schwarz inequality states that for all vectors $u$ and $v$ of an inner product space, it is true that
  \begin{align*}
    |<u, v>|^{2} &\leq <u,u> \cdot <v,v>& \\
    |<x, \mathbf{1}>|^2 &\leq <x, x> \cdot <\mathbf{1}, \mathbf{1}>& \\
    ||x||_{1}^{2} &\leq ||x||_{2}^{2} \cdot n& \\
    ||x||_{1} &\leq \sqrt{n}||x||_{2}&
  \end{align*}

  Next, we will prove that $||x||_{\infty} \leq ||x||_{1}$. We know that $||x||_{\infty} = \text{max}(x)$ and $||x||_{1} = \sum_{i=1}^{n}|x_i|$, which includes the max and is, by default, at least as large as the L-infinity norm.
\end{itemize}

\end{document}

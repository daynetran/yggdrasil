% Search for all the places that say "PUT SOMETHING HERE".

\documentclass[11pt]{article}
\usepackage{amsmath,textcomp,amssymb,geometry,graphicx,enumerate}

\def\Name{Dayne Tran}  % Your name
\def\SID{3034839991}  % Your student ID number
\def\Homework{00} % Number of Homework
\def\Session{Spring 2022 }


\title{CS170--Spring 2022 --- Homework \Homework}
\author{\Name, SID \SID}
\markboth{CS170--\Session\  Homework \Homework\ \Name}{CS170--\Session\ Homework \Homework\ \Name}
\pagestyle{myheadings}
\date{}

\newenvironment{qparts}{\begin{enumerate}[{(}a{)}]}{\end{enumerate}}
\def\endproofmark{$\Box$}
\newenvironment{proof}{\par{\bf Proof}:}{\endproofmark\smallskip}

\textheight=9in
\textwidth=6.5in
\topmargin=-.75in
\oddsidemargin=0.25in
\evensidemargin=0.25in


\begin{document}
\maketitle

Collaborators: None

\section*{1. Course Syllabus}
\begin{qparts}
\item
None.

\item
Yes, but anonymously.
\end{qparts}



\newpage
\section*{2.}
\begin{qparts}
\item
Midterm 1 is on February 23, 8pm-10pm. Midterm 2 is on April 5, 8pm-10pm. Final is on May 11, 11:30am-2:30pm.

\item
The course staff recommend having the homework finished by 10pm.

\item
``[Course Staff] accept absolutely no submissions after 11:59pm, even after technical issues or emergencies. No exceptions.''

\item
Piazza

\item
I have read and understood the course syllabus and policies.

\end{qparts}


\newpage
\section*{3.}
\begin{qparts}
\item
Not OK
\item
Not OK
\item
Not OK
\item
Not OK
\end{qparts}


\newpage
\section*{4.}
Refer to Course Notes page 3, where it is shown that applying limit tests to f and g asymptotics can be used for proofs. Rough draft: define f(n) as $2^n$.\\
First, we want to prove that for all $c>0$, $f(n)=\Omega(n^c)$, i.e. $n^c$ dominates $f(n)$. We know from lectures and course notes that we can use limit tests to prove dominance.
\begin{align*}
    &{(1^+)}^n = \Omega(n^c) \Longleftrightarrow 
    \lim_{n \to \infty} \frac{(1^+)^n}{n^c} > 0 \text{ for all } c > 0& \\
    \intertext{Apply L'Hopital's Rule because the immediate result is $\frac{\infty}{\infty}$:}
    &\lim_{n \to \infty} \frac{\ln 1^+ \cdot {(1^+)}^n}{c \cdot n^{c-1}}&\\
    \intertext{The result is still $\frac{\infty}{\infty}$, so apply L'Hopital's Rule repeatedly until the following expression appears:}
    &\lim_{n \to \infty} \frac{{(\ln 1^+)}^{c} \cdot {(1^+)}^n}{c!} = \infty > 0&
\end{align*}
Second, we want to prove that for all $\alpha > 1$, $f = O(\alpha ^n)$. Again, using our lecture and course notes, we can make use of limit tests for our proof.
\begin{align*}
    {(1^+)}^n &= O(\alpha^n) \Longleftrightarrow \lim_{n \to \infty} \frac{{(1^+)}^n}{\alpha^n} < \infty \text{ for all } \alpha > 1&\\
    \intertext{We can simplify the limit to get to a determinate result:}
    \lim_{n \to \infty} \frac{{(1^+)}^n}{\alpha^n} &= \lim_{n \to \infty} \left(\frac{{1^+ \cdot 1^+ \cdot \ldots \cdot 1^+}}{\alpha \cdot \alpha \cdot \ldots \alpha}\right)&\\
    &= \lim_{n \to \infty} \left(\frac{1^+}{\alpha} \cdot \frac{1^+}{\alpha} \cdot \ldots \cdot \frac{1^+}{\alpha}\right)&\\
    &= \lim_{n \to \infty} {\left(\frac{1^+}{\alpha}\right)}^n&\\
    &\leq 1 \text{ since $\frac{1^+}{\alpha} \leq 1$ for $\alpha > 1$.}&\\
    &<\infty
\end{align*}

\newpage
\section*{5.}
We want to prove that for any $\epsilon > 0$, we have log $x \in O(x^\epsilon)$.
\begin{align*}
    log(x) &= O(x^\epsilon) \Longleftrightarrow \lim_{x \to \infty} \frac{log(x)}{x^\epsilon} < \infty \text{ for all } \epsilon > 0&\\
    \intertext{Apply L'Hopital's Rule because the immediate result is $\frac{\infty}{\infty}$.}
    & = \lim_{x \to \infty} \frac{1}{x} \cdot \frac{1}{\epsilon \cdot x^{\epsilon - 1}}& \\
    \intertext{Plug in $0^+$ for $\epsilon$ since $\epsilon > 0$.}
    & = \lim_{x \to \infty} \frac{1}{0^{+}x^{0^+}} \\
    &= \frac{1}{\infty} = 0 < \infty \\
    &\Longleftrightarrow log(x) = O(x^\epsilon) \text{for all $\epsilon > 0$.}&
\end{align*}
\end{document}